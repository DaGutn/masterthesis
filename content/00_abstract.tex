\thispagestyle{plain}

\section*{Abstract}
    Antiferromagnetic van der Waals materials have gained a lot of attention in recent years.
    The possibility to be exfoliated down to a monolayer, combined with their antiferromagnetic properties
    % and incorperated into 2D heterostructures
    % of being insensivite to external magnetic fields and they featuring spin dynamics in the THz range
    , makes them promising candidates for application in future spintronic devices. 
    However, to achieve this, a detailed knowledge of their electronic band structure is crucial. Here, we investigate one of the most studied 2D antiferromagnets, transition metal phosphorous trisulfides (MPS$_\text{3}$), by means of photoemission spectroscopy (PES). 
    By utalizing angle-resolved PES the valence band structure of MPS$_\text{3}$ (M=Fe,Ni,Co,Mn) is mapped in momentum space and compared to DFT calculations.
    % With the help of DFT calculations we identify bands originating from different atomic orbitals, which allow us to describe the occuring differences in the dispersion between these materials.
    By adsorbing thin alkali metal films on CoPS$_\text{3}$ and FePS$_\text{3}$,
    we are able to reduce the insulating properties and thus enable measurements at low temperatures.
    % we are able to reduce the meterials insulating properties und  overcome the charging problem at low temperatures.
    Thereby electrons stemming from the adsorbed alkali metal atoms are occupying new bands immediately above the valence band maximum, turning the crystal surface from a semiconducting state to a metallic one.
    Lastly, by using time-resolved PES, we observe the characteristic fingerprint of intra-ionic d-d transitions in the Fe$^{+2}$ ion of FePS$_\text{3}$ and study their temporal evolution.
    % These finding represent a step     
\section*{Kurzfassung}
\begin{foreignlanguage}{ngerman}
    Antiferromagnetische van-der-Waals-Materialien haben in den letzten Jahren viel Aufmerksamkeit erregt.
    Die Möglichkeit, sie bis auf eine Monolage zu exfolieren, macht sie in Verbindung mit ihren antiferromagnetischen Eigenschaften zu vielversprechenden Kandidaten für den Einsatz in künftigen spintronischen Geräten. Um dies zu erreichen, ist jedoch eine detaillierte Kenntnis ihrer elektronischen Bandstruktur entscheidend. Hier untersuchen wir einen der meist untersuchten 2D-Antiferromagneten, die Übergangsmetall-Phosphor-Trisulfide (MPS$_\text{3}$), mit Hilfe der Photoelektronenspektroskopie (PES). 
    Mit dem Einsatz von winkelaufgelöster PES wird die Valenzbandstruktur von MPS$_\text{3}$ (M=Fe,Ni,Co,Mn) im Impulsraum abgebildet und mit DFT-Berechnungen verglichen.
    Durch die Adsorption dünner Alkalimetallfilme auf CoPS$_\text{3}$ und FePS$_\text{3}$ sind wir in der Lage, die isolierenden Eigenschaften zu reduzieren und somit Messungen bei niedrigen Temperaturen zu ermöglichen.
    Dabei besetzen die von den adsorbierten Alkalimetallatomen stammenden Elektronen neue Bänder unmittelbar oberhalb des Valenzbandmaximums, wodurch die Kristalloberfläche von einem halbleitenden Zustand in einen metallischen übergeht.
    Schließlich beobachten wir mittels zeitaufgelöster PES den charakteristischen Fingerabdruck intra-ionischer d-d-Übergänge im Fe$^{+2}$-Ion von FePS$_\text{3}$ und untersuchen ihre zeitliche Entwicklung.
\end{foreignlanguage}