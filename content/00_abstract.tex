\thispagestyle{plain}

\section*{Abstract}
A distinguishing feature of antiferromagnets is their exceptionally high AFMR frequencies in the THz-range.
This was subject of an extensive amount of studies especially since the advent of femtosecond laser technologies which ushered in a new era of understanding and manipulation of ultrafast spin dynamics.
This work is a preliminary study aiming to understand the interplay between organic molecules and AFM properties, potentially modifying magnon frequencies and amplitudes.
Employing femtosecond pump-probe spectroscopy enables us to examine their ultrafast magnon dynamics of the Ni-based compounds.
Using NiO as a prototypical AFM, we cover it with C60 and focus on the amplification of magnon amplitudes by pumping around d-d transitions.
The C60 molecule coverage yields no change in frequency but a shift in resonance hints at an influnce on the magnon dynamics. 
Pumping off-resonantly reveals a yet unreported mode at \qty{0.62}{THz}. 
% This thesis explores the influence of organic molecules, specifically C60, on magnon characteristics at the NiO/C60 interface.
Moving further into the nanoscale, our investigation extends to quasi-2D AFMs, exemplified by the van-der-Waals material NiPS3.
% Pumping d-d transitions in 2D compounds opens up a direct path to manipulate magnetic anisotropy.
Two hitherto unreported modes are found here at 13 and \qty{28}{GHz}.
The combined investigation of temperature and polarization dependencies reveals that a coupling between the lattice and the magnetic system is involved in their excitation mechanism.
% leaving their exact origin to be determined by further studies.













% \section*{Kurzfassung}
% \begin{foreignlanguage}{ngerman}
%     Antiferromagnetische van-der-Waals-Materialien haben in den letzten Jahren viel Aufmerksamkeit erregt.
%     Die Möglichkeit, sie bis auf eine Monolage zu exfolieren, macht sie in Verbindung mit ihren antiferromagnetischen Eigenschaften zu vielversprechenden Kandidaten für den Einsatz in künftigen spintronischen Geräten. Um dies zu erreichen, ist jedoch eine detaillierte Kenntnis ihrer elektronischen Bandstruktur entscheidend. Hier untersuchen wir einen der meist untersuchten 2D-Antiferromagneten, die Übergangsmetall-Phosphor-Trisulfide (MPS$_\text{3}$), mit Hilfe der Photoelektronenspektroskopie (PES). 
%     Mit dem Einsatz von winkelaufgelöster PES wird die Valenzbandstruktur von MPS$_\text{3}$ (M=Fe,Ni,Co,Mn) im Impulsraum abgebildet und mit DFT-Berechnungen verglichen.
%     Durch die Adsorption dünner Alkalimetallfilme auf CoPS$_\text{3}$ und FePS$_\text{3}$ sind wir in der Lage, die isolierenden Eigenschaften zu reduzieren und somit Messungen bei niedrigen Temperaturen zu ermöglichen.
%     Dabei besetzen die von den adsorbierten Alkalimetallatomen stammenden Elektronen neue Bänder unmittelbar oberhalb des Valenzbandmaximums, wodurch die Kristalloberfläche von einem halbleitenden Zustand in einen metallischen übergeht.
%     Schließlich beobachten wir mittels zeitaufgelöster PES den charakteristischen Fingerabdruck intra-ionischer d-d-Übergänge im Fe$^{+2}$-Ion von FePS$_\text{3}$ und untersuchen ihre zeitliche Entwicklung.
% \end{foreignlanguage}