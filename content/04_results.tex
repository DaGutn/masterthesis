\chapter{Results and discussion}
\section{Examplary analysis of pump-probe trace}

\section{NiO/C60}
\subsection{Absorption measurements on NiO/C60}
NiO is used as a prototypical antiferromagnet 
But for us it was particular interesting what changes we could detect in the magnetic behavior by covering the NiO with a molecule layer.
To try this, C60 was used in part because of its availability to our group, but mainly because of its strong interaction with ferromagnetic surfaces / strong dipole formation when in contact with () .....
\begin{figure}[ht]
    \centering
    \includegraphics[width=0.7\textwidth]{pictures/plots/NiO_absorption_nm.pdf}
    \caption{Showing the absorption spectra measured on the CARY system for pure NiO (blue line) and for NiO/C60 (red line).}
    \label{fig:NiO_absorption_nm}
\end{figure}
\FloatBarrier
As the samples did not have the same size, a direct comparison of the two spectra by subtraction or division was not possible.
Nonetheless, an obvious broadening of the exciton peak at \qty{1280}{nm} and the magnon-sideband at \qty{1273}{nm} is visible in \autoref{fig:NiO_absorption_nm}, especially in the zoomed-in view on the bottom.
While still being distinct the peaks seem to have shifted slightly to higher energies by about \qty{1}{meV}.
Furthermore some adidtional features between \qty{1350}{nm} and \qty{1400}{nm} are clearly visible in the spectrum of the NiO/C60 in contrast to the one from the pure NiO and may be worthwhile to examine.

\subsection{Pump wavelenth dependence of NiO/C60}
A pump wavelength dependence on pure NiO had already been done by Bossini et al. \lit{Bossini et al.} which is also featured here.
Now we wanted to see what impact the hybridization of the pure NiO surface with the C60-molecule would have on the domain structure and thus on the coupling of the two magnon modes.

\begin{figure}[ht]
    \centering
    \includegraphics[width=0.7\textwidth]{pictures/plots/NiO_pump_WL_dependence.png}
    \caption{The time traces taken by pump-probe spectroscopy for different pump wavelength ranging from \qty{1158}{nm} to \qty{1362}{nm} are shown here. As the two oscillations are so pronounced in these plots an analysis by fitting the data was chosen. The red line depicts the fit of the lf-mode together with the background.}
    \label{fig:NiO_pump_WL_dependence}
\end{figure}

\begin{figure}[ht]
    \centering
    \includegraphics[width=0.7\textwidth]{pictures/plots/NiO_pump_WL_dependence_THz.png}
    \caption{The measured data after subtracting the fit of the lf-mode from it. The red line shows the fit of the hf-mode. The two upper traces of }
    \label{fig:NiO_pump_WL_dependence_THz}
\end{figure}
\begin{figure}[ht]
    \centering
    \includegraphics[width=0.6\textwidth]{pictures/plots/NiO_pump_WL_dependence_freq.png}
    \caption{The time traces taken by pump-probe spectroscopy for different pump wavelength ranging from \qty{1158}{nm} to \qty{1362}{nm}.}
    \label{fig:NiO_pump_WL_dependence_freq}
\end{figure}
\begin{figure}[ht]
    \centering
    \includegraphics[width=0.6  \textwidth]{pictures/plots/NiO_pump_WL_dependence_ampl.png}
    \caption{The time traces taken by pump-probe spectroscopy for different pump wavelength ranging from \qty{1158}{nm} to \qty{1362}{nm}.}
    \label{fig:NiO_pump_WL_dependence_ampl}
\end{figure}



\section{NiPS3}