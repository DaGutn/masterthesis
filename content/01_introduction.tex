\chapter{Introduction}
Since the advent of spintronics understanding and control of the time evolution of magnetic systems grew to be an important research field in modern solid state physics.
The discovery of the exchange bias in 1956 \cite{meiklejohn_new_1957} sparked first interest in antiferromagnets (AFM), which was further ignited by the discovery of the giant magnetoresistance effect in antiferromagnetically coupled layered magnetic systems in 1988 \cite{binasch_enhanced_1989, camley_theory_1989}.
Manipulation of the magnetic order in AFM directly however remained inaccessible, as the magnetic field needed to reorient spins scales with the exchange interaction and thus reaches values around \qty{100}{T} \cite{nemec_antiferromagnetic_2018}.

Furthermore, AFM posses no or close to no external magnetic stray fields, making them particularly practical for usage in scaled-down devices as they do not influence other devices even in extreme proximity.
At the same time, this represents the crux of the difficulties one is facing when trying to observe the dynamics of the antiferromagnetic order parameter.

It is accepted that the fastest way to influence the magnetization orientation is via spin precession by applying an external magnetic field.
For FM this field has to be applied until the magnetization crosses the potential barrier from one metastable state to another.
AFM on the other hand exhibit something akin to intertia driven motion which makes it possible to excite precessional motion of the magnetization with stimuli shorter than the duration of overcoming the potential barrier \cite{kimel_inertia-driven_2009}.

In contrast to ferromagnets (FM), which have collective spin excitations in the GHz-range, AFM have much higher antiferromagnetic resonance (AFMR) frequencies up to the THz-range.

Femtosecond lasers are an important tool making research on AMF accessible and addressing all the aforementioned issues.
Even in AFM an ultrashort pulse induces an ultrashort effective magnetic field via Raman-like mechanisms such as the inverse Faraday or the inverse Cotton-Mouton effect.
Detection is also based on the magneto-optical effects.
Rotation of polarization or ellipticity is induced by the magnetization which is generated by the spin precession even in compensated AFM\cite{nemec_antiferromagnetic_2018, bossini_femtosecond_2017}.
Using a femtosecond laser for pump-probe spectroscopy it is possible to resolve even the high frequencies of the AFMR.

In the past, optical magnetization control has been limited to a binary process.
In the context of data storage that means that a laser pulse reads or writes a magnetic bit exhibiting a positive or negative magnetization.
Later, it was shown that full vectorial control of the magnetization state is possible \cite{kanda_vectorial_2011, satoh_writing_2015}.
Addressing e.g. the phase and the amplitude of magnon modes independently by laser pulses, multiple pieces of information can be stored in a single storage element \cite{nemec_antiferromagnetic_2018}.

In this work we concentrate on the amplification of the magnon amplitude by pumping d-d transitions.
The quenching of orbital angular momentum of transition metal ions in a crystal field leads to no spin-orbit coupling in the ground state.
So that only higher-energy orbital states are unquenched.
Large anisotropy is tightly linked to unquenched orbital angular moments.
Meaning that pumping of these higher orbital states represents a direct way of manipulating the magnetic anisotropy and thus the AFMR.
For transition metals both the ground state and higher-energy states are d-states, which gives these transition its name.

In this regard, NiO is used as a prototypical antiferromagnet to expand the fundamental understanding of ultrafast spin dynamics in AFM, as it features a high Neél-temperature, a well-known AFM order and a high AFMR frequency.
Despite being known as an AFM for decades, some questions about the properties of NiO remain not fully resolved, like the significance of the dipolar interaction in the two-sublattice model or the compatibility of the two- and eight-sublattice model in describing the magnetic-field dependence of AFMR-modes \cite{ohmichi_frequency-domain_2022}.

As organic molecules are extremely versatile their properties can be finely tuned \cite{sanvito_molecular_2011}.
The pump mechanism relies on inducing a transient magneto-crystalline anisotropy.
C$_{60}$ molecules are known for creating interfacial dipoles \lit{Veenstra} and enhancing the interfacial magnetic anisotropy in FM \lit{Bairagi}.
So that the an interface between NiO and C$_{60}$ will surely have an influence on the magnetic anisotropy and through this on the AFMR.

Thin AFM, scalable down to the nanometer scale with their magnetic properties intact, are promising for spintronic application.
Going one step further, magnetic van-der-Waals materials such as transition metal thiophosphates ($M$PS3) are especially interesting in that context, as they retain their magnetic order down to a few monolayers.
Here, we study NiPS3.
It was chosen due to its similar electronic structure with respect to NiO, as the orbital angular momentum of its ground state is also quenched.
Large anisotropy is usually linked to unquenched orbital angular moments.
Moreover the Mermin-Wagner theorem states, 2D magnetic systems without anisotropy cannot develop magnetic order.
Thus, the magnetic anisotropy and also the magnetic order in 2D systems orginates from the mixing of the groundstate with higher orbital states.
Pumping the d-d transitions, meaning the excitation of the higher orbital states, represents a direct path to controling the magnetic anisotropy \cite{afanasiev_controlling_2021}.


(the more indirect the pump and detection mechanisms get, the more difficult it gets to seperate the contributions to the signal associated with the magnetic and the lattice system \cite{fiebig_ultrafast_2008}) \\
(manipulation of spins requires exceedingly high magnetic fields, which can be supplied since the advent of femtosecond laser pulses?)

