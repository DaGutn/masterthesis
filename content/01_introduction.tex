\chapter{Introduction}
Since the advent of ultrafast laser technologies understanding and controlling the time evolution of magnetic systems has become an important area of research for development of ultrafast spintronic devices.
The discovery of the exchange bias effect in 1956 \cite{meiklejohn_new_1957} sparked the first interest in employing antiferromagnets (AFM) for applications, which in turn led to the discovery of the giant magnetoresistance effect in antiferromagnetically coupled layered magnetic systems in 1988 \cite{binasch_enhanced_1989, camley_theory_1989}.
Direct manipulation of the magnetic order in AFM directly however remained inaccessible, as the magnetic field needed to reorient spins scales with the exchange interaction and thus reaches values around \qty{100}{T} which is much larger than that of ferromagnets (FM) \cite{nemec_antiferromagnetic_2018}.
In contrast to FM, which have collective spin excitations in the GHz-range, AFM have much higher antiferromagnetic resonance (AFMR) frequencies up to the THz-range.
Furthermore, AFM posses no external magnetic stray fields, making them particularly practical for usage in scaled-down devices as they do not influence other devices even in extreme proximity.
At the same time, this is the crux of the difficulties encountered when trying to study the dynamics of the antiferromagnetic order parameter.

% It is accepted that the fastest way to influence the magnetization orientation is via spin precession by applying an external magnetic field in the form of ultrashort laser pulses.
% For FM this field has to be applied until the magnetization crosses the potential barrier from one metastable state to another.
% AFM on the other hand exhibit something akin to intertia driven motion which makes it possible to excite precessional motion of the magnetization with stimuli shorter than the duration of overcoming the potential barrier \cite{kimel_inertia-driven_2009}.

Femtosecond lasers are an important tool making research on AFM accessible and addressing the aforementioned issues.
Even in AFM, an ultrashort laser pulse is capable to perturb the magnetic order and excite the spin system.
% Even in AFM, an ultrashort laser pulse induces an ultrashort effective magnetic field via Raman-like mechanisms such as the inverse Faraday or the inverse Cotton-Mouton effect.
In addition to the excitation, fs- laser pulses allow for an ultrafast detection based on the magneto-optical effects.
Rotation of polarization or ellipticity is induced by the magnetization which is generated by the spin precession even in compensated AFM\cite{nemec_antiferromagnetic_2018, bossini_femtosecond_2017}.
% Rotation of polarization or ellipticity is induced by the magnetization which is generated by the spin precession even in compensated AFM\cite{nemec_antiferromagnetic_2018, bossini_femtosecond_2017}.
By employing a femtosecond laser in pump-probe spectroscopy, it is possible to resolve even the THz frequencies of the AFMR. \\
Ultrafast excitation and detection of high-frequency AFM magnons would also change the paradigm of optical magnetization control which has been limited to a binary process in the past.
In the context of data storage that meant that a laser pulse would read or write a magnetic bit either exhibiting a positive or a negative magnetization.
Later, it was shown that full vectorial control of the magnetization state is possible \cite{kanda_vectorial_2011, satoh_writing_2015}.
Addressing e.g. the phase and the amplitude of magnon modes independently by ultrashort laser pulses, multiple pieces of information can be stored in a single storage element \cite{nemec_antiferromagnetic_2018}.

In this work we concentrate on the amplification of the magnon amplitude by pumping the d-d transitions in Ni-based antiferromagnetic compounds.
% The quenching of orbital angular momentum of transition metal ions in a crystal field leads to no spin-orbit coupling in the ground state.
% So that only higher-energy orbital states are unquenched.
% Large anisotropy is tightly linked to unquenched orbital angular moments.
% Meaning that pumping of these higher orbital states represents a direct way of manipulating the magnetic anisotropy and thus the AFMR.
% For transition metals both the ground state and higher-energy states are d-states, which gives these transition its name.
In this regard, NiO is used as a prototypical antiferromagnet to expand the fundamental understanding of ultrafast spin dynamics in AFM, as it features a high Neél-temperature (in bulk $T_N=\qty{523}{K}$), a well-known AFM order and a high AFMR frequency \cite{rezende_introduction_2019}.
Despite being known as an AFM for decades, some questions about the properties of NiO remain not fully resolved, like the compatibility of the two- and eight-sublattice model in describing the magnetic-field dependence of AFMR-modes \cite{ohmichi_frequency-domain_2022}. \\
So far a theoretical description as well as a comprehensive investigation of the interface between organic molecules and antiferromagnetic materials are missing.
Our work is a preliminary attempt to investigate organic molecule induced modifications in the magnon characteristics, such as frequency and amplitude.
As organic molecules are extremely versatile their properties can be finely tuned \cite{sanvito_molecular_2011}.
% The pump mechanism relies on inducing a transient magneto-crystalline anisotropy.
C60 molecules are known for creating interfacial dipoles \cite{veenstra_interface_2002} and enhancing the interfacial magnetic anisotropy in FM \cite{bairagi_tuning_2015}.
AFMR frequencies depend on the magnetic anisotropy \cite{rezende_introduction_2019}.
So we expect the NiO/C60 interface to have an influence on the magnetic anisotropy of NiO and thus on its AFMR frequencies.

Thin AFMs, scalable down to the nanometer scale with their magnetic properties intact, are promising for spintronic applications.
Going one step further, magnetic van-der-Waals materials such as transition metal thiophosphates ($M$PS3) are particularly interesting in this context, as they retain their magnetic order down to a few monolayers \cite{kim_suppression_2019}.
Here, we study NiPS3.
It was chosen because of its similar electronic structure with respect to NiO, as the orbital angular momentum of its ground state is also quenched by the crystal field \cite{ogale_functional_2013}.
Only the higher-energy states remain unquenched.
Large anisotropy is usually linked to unquenched orbital angular moments \cite{afanasiev_controlling_2021}.
Moreover, the Mermin-Wagner theorem states that 2D magnetic systems without anisotropy cannot develop magnetic order.
Thus, the magnetic anisotropy and also the magnetic order in 2D systems originate from the mixing of the groundstate with higher orbital states.
Pumping the d-d transitions, meaning the excitation of the higher orbital states, represents a direct path to controlling the magnetic anisotropy \cite{afanasiev_controlling_2021}.


% (the more indirect the pump and detection mechanisms get, the more difficult it gets to seperate the contributions to the signal associated with the magnetic and the lattice system \cite{fiebig_ultrafast_2008}) \\
% (manipulation of spins requires exceedingly high magnetic fields, which can be supplied since the advent of femtosecond laser pulses?)

