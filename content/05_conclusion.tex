\chapter{Conclusion and Outlook}

In this thesis, pump-probe spectroscopy measurements were carried out on a \qty{15}{\um}-thick NiO crystal with adsorbed C60 molecules as well as on a NiPS3 bulk crystal.
These Ni-based compounds were chosen because they are both AFM, feature a similar d-d transition and thus exhibit tuning potential.
The femtosecond laser system used makes it possible to tune the photon energies of both pump and probe beam separately, covering the energy range from \qty{0.5}{eV} to \qty{3.5}{eV}.
The samples are mounted in a cryostat which allows them to be cooled down to \qty{77}{K}.
It is also possible to conduct polarization dependent measurements.
% Combinations of polarizators and $\frac{\lambda}{2}$-waveplates are also included in the experimental setup to conduct polarization dependent measurements while maintaining the same fluence.
% Using these capabilities to their fullest, the measurements were taken that yield the following results.

First, measurements on NiO/C60 were performed in order to investigate the two magnon modes at 0.13 and \qty{1.07}{THz}.
% First, the results pertaining to the differences measured between NiO/C60 and pure NiO are summarized.
The absorption spectra taken on a commercial CARY system exhibit a X-M peak.
A dependence on the pump wavelength was acquired for NiO/C60 to be able to compare it to the previously measured spectral dependence of pure NiO.
A fit of the obtained data traces shows no change in the frequency of any of the two magnon modes meaning that the adsorbed C60 does not induce a change in the interfacial magnetic anisotropy.
On the other hand, a slight shift of the amplification of both magnon modes in the spectral dependence when pumping around the X-M peak could be detected.
A distinct sharpening of the amplification peak is visible as well.
From an absorption measurement done on a commercial CARY system some features at \qty{1400}{nm} could be seen in the spectrum NiO/C60, distinguishing it from the spectrum of pure NiO.
Subsequent measurements at \qty{1400}{nm} report a hitherto unreported mode of \qty{0.62}{THz}.

Concerning NiPS3, we first tried to excite the magnon modes, which were e.g. found by Afanasiev et al., by pumping around \qty{1.1}{eV}.
% The measurements on NiPS3 started with the initial intention of reproducing the results from Afanasiev et al..
Unfortunately these magnons could not be detected.
However, two modes at 13 and at \qty{28}{GHz} were extracted via FFT of the rotation of polarization which led to a concentration of the research efforts on these modes.
During all dependencies the frequencies of the two modes stayed the same.
In the transient reflectivity only the \qty{28}{GHz}-mode could be detected hinting at the fact that the origin of this mode are lattice dynamics.
To understand the underlying excitation mechanism, a pump polarization dependence was done which yielded a 3-fold symmetry pointing to a connection of the excited mode to the hexagonal structure of the Ni$^{2+}$-ions.
This dependence was visible in both the rotation of the polarization and the transient reflectivity.
In an effort to determine the detection mechanism, a probe polarization dependence was measured.
It revealed a 2-fold symmetry in the rotation of polarization which can be traced back to a linear birefringence or the Cotton-Mouton effect as opposed to the Kerr effect which would not exhibit any dependence on the probe polarization angle.
The Cotton-Mouton effect appears as a consequence of an in-plane oscillation of the Néel vector.
This leads to the conclusion that magnetic dynamics are excited in NiPS3.
A temperature dependence was taken to determine the type of origin of both modes.
The amplitudes of the \qty{13}{GHz}-oscillations in the rotation of polarization vanish above the Néel temperature of NiPS3 at \qty{155}{K} which suggests a magnetic origin.
The \qty{28}{GHz}-oscillations in the rotation of polarization vanish at \qty{120}{K}, whereas the behavior of the amplitudes in the transient reflectivity is completely random.
That the \qty{28}{GHz}-mode persists in the transient reflectivity even after the Néel temperature reinforces the assumption that it emerges from the lattice.
The mode being visible in the rotation of polarization and vanishing before reaching the Néel temperature indicates that it also has a magnetic component.
As a conclusion, it can be said that the \qty{28}{GHz}-mode is non-magnetic but couples to the magnetic system.

Although the measurements on NiO/C60 could reproduce the literature and expand on it, they also lead to some open questions.
The measurements of the \qty{0.62}{THz}-mode were done after the absorption measurements and thus also after the reduction of the sample quality.
So it could be worthwhile to repeat these measurements on a newly prepared crystal.
The identification of the \qty{0.62}{THz}-mode in NiO/C60 can be subject of further investigations.
A magnetic field dependence would be reasonable, as an applied field could align more domains resulting in a stronger signal. \\
Measuring the magnetic dependence of the two modes found in NiPS3 would contribute to a definitive assignment of the origin of these modes, i.e. whether they are magnetic in nature or not.
A spectral dependence similar to the one conducted for NiO/C60 could also help in identifying the excitation mechanism of the modes.
Using this, the connection between the modes and the pumped d-d transition could be examined.

% The meaning behind it being
% raises the question of



